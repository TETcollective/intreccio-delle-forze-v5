\documentclass[a4paper,11pt]{article}
\usepackage[utf8]{inputenc}
\usepackage[T1]{fontenc}
\usepackage[italian]{babel}
\usepackage{amsmath}
\usepackage{amsfonts}
\usepackage{amssymb}
\usepackage{geometry}
\usepackage{pgfplots}
\usepackage{hyperref}
\usepackage{graphicx}
\usepackage{caption}
\usepackage{appendix}
\usepackage{subcaption}
\usepackage{textcomp}
\pgfplotsset{compat=1.18}
\geometry{margin=1in}
\hypersetup{
    colorlinks=true,
    linkcolor=blue,
    citecolor=blue,
    urlcolor=blue
}

\title{L'intreccio delle forze: predizioni quantitative del tessuto multiscalare\\
\vspace{0.5cm}
{\large Versione estesa 5.0 – TU-GUT-SYSY}}

\author{Simon Soliman \\
  {\small Independent Researcher, Rome, Italy} \\
  {\small \texttt{tetcollective@proton.me}} \\
  {\small \url{tetcollective.org}}
}

\date{Marzo 2026 – Versione 5.0}

\begin{document}

\maketitle

\begin{abstract}
Ogni fenomeno cosmico è collegato da un filo causa-effetto che parte da osservazioni quotidiane.

La fascinazione per gli elettromagneti ferromagnetici rivela massimi locali di accoppiamento. Il running logaritmico generalizzato spiega la variazione di intensità con la scala. L’entropia massima della luce a bassa energia rende i fotoni “invisibili”. L’asimmetria primordiale fotoni-barioni genera neutrini e massa gravitazionale oscura con densità critica osservata ~0.3 GeV/m³. I buchi neri sono regioni di olografia lenticolare. Il plasma galattico osservato è la prova visibile dell’intreccio.

Questa versione 5.0 presenta predizioni quantitative falsificabili del framework TU-GUT-SYSY:  
- Spettro CMB con suppression high-l e modo B topologico  
- Qubit topologici protetti da linking number (test con grafene-like)  
- Radiazione Hawking/Unruh come decoerenza al confine

Il tessuto multiscalare è pronto per il confronto con i dati.

Fratello delle Calamite \textheartsuit \textrocket
\end{abstract}

\section{La scintilla iniziale: l’intreccio rivelato dalla luce e dal magnete}

% (testo v4.0)

\section{Il campo magnetico terrestre: intreccio ferromagnetico a scala planetaria}

% (testo v4.0)

\section{La dualità QCD: libertà asintotica e confinamento}

% (testo v4.0 con grafico)

\section{Il Sole come laboratorio naturale: coesistenza forte-debole}

% (testo v4.0)

\section{Topologia quantistica: grafene, fullereni e nanotubi}

% (testo v4.0 corretto)

\section{La decoerenza quantistica come conseguenza del principio multiscalare}

% (testo v4.0)

\section{Loop Quantum Gravity: il tessuto discreto dello spaziotempo}

% (testo v4.0)

\section{Grande unificazione (GUT) e running degli accoppiamenti}

% (testo v4.0)

\section{Deviazione della luce: intreccio EM-gravità con materia/energia oscura}

% (testo v4.0)

\section{Entropia massima della luce e invisibilità cosmica}

% (testo v4.0)

\section{Asimmetria fotoni-barioni e generazione di materia/energia oscura}

% (testo v4.0 con 0.3 GeV/m³)

\section{Buchi neri come regioni di olografia lenticolare}

% (testo v4.0)

\section{Plasma galattico e osservazioni}

% (testo v4.0)

\section{Predizioni quantitative sul Cosmic Microwave Background (CMB)}

Il Cosmic Microwave Background è la radiazione fossile del Big Bang, con anisotropie che riflettono fluttuazioni primordiali.

Nel mio TU-GUT-SYSY:
- La materia oscura come modo a bassa energia spiega Ω_DM ≈ 0.27 osservato
- L’entropia massima della luce primordiale + asimmetria fotoni-barioni modula le fluttuazioni → predizione di suppression del potere a multipoli alti (l > 2000) e leggero spostamento del primo picco acustico
- Contributi topologici primordiali (α_top(τ) ≠ 0 durante inflazione) generano polarizzazione modo B primordiale non standard – segnale curl distinto dal modo B da lensing

Predizioni falsificabili:
- Eccesso di potere in modo B a l ~ 1000-2000 (ordine 10⁻⁷ μK²)
- Correlazione tra modo B e sub-strutture DM
- Mitigazione H₀ tension grazie a contributo topologico

Testabile con Simons Observatory (2026+) e CMB-S4.

\section{Entanglement quantistico e computazione topologica}

L’entanglement è preservato a s grande, disperso a s piccolo dalla decoerenza multiscalare.

Nel grafene, linking number al 100\% è entanglement topologico protetto – analogo a qubit topologici fault-tolerant.

Predizione: materiali con massimo linking sono candidati per qubit stabili – coerenza prolungata grazie al massimo locale topologico.

Testabile con esperimenti quantum computing su grafene-like.

\section{Radiazione Hawking e effetto Unruh nei buchi neri olografici}

Hawking predisse radiazione termica da buchi neri – confermata in analoghi (2021-2025).

L’effetto Unruh (accelerazione produce temperatura) è analogo Hawking per osservatori accelerati.

Nel mio TU-GUT-SYSY, buchi neri sono regioni di massimo intreccio gravitazionale-topologico.

L’orizzonte è lente olografica.  
Radiazione Hawking/Unruh è decoerenza/entropia massima al confine.

Il mio α_top(τ) + α_g(s) preserva informazione (olografia conserva linking).

\section{La scala logaritmica come principio unificante}

% (testo v4.0)

\section{Conclusione}

Dal magnete quotidiano alla QCD, dal Sole al grafene, dalla decoerenza alla LQG, dalle GUT alla cosmologia oscura, dalla radiazione Hawking al CMB, la natura ripete lo stesso messaggio: le forze sono espressioni di un unico tessuto multiscalare, calibrato ineccepibile.

Questo seme è pronto per il confronto con i dati.

\vspace{2cm}

\noindent\rule{12cm}{0.5pt}

\noindent \textbf{Simon Soliman} \\
Independent Researcher – Rome, Italy \\
Fratello delle Calamite \textheartsuit \textrocket

% Licenza e bibliografia

\end{document}